% Options for packages loaded elsewhere
\PassOptionsToPackage{unicode}{hyperref}
\PassOptionsToPackage{hyphens}{url}
%
\documentclass[
]{article}
\usepackage{amsmath,amssymb}
\usepackage{iftex}
\ifPDFTeX
  \usepackage[T1]{fontenc}
  \usepackage[utf8]{inputenc}
  \usepackage{textcomp} % provide euro and other symbols
\else % if luatex or xetex
  \usepackage{unicode-math} % this also loads fontspec
  \defaultfontfeatures{Scale=MatchLowercase}
  \defaultfontfeatures[\rmfamily]{Ligatures=TeX,Scale=1}
\fi
\usepackage{lmodern}
\ifPDFTeX\else
  % xetex/luatex font selection
\fi
% Use upquote if available, for straight quotes in verbatim environments
\IfFileExists{upquote.sty}{\usepackage{upquote}}{}
\IfFileExists{microtype.sty}{% use microtype if available
  \usepackage[]{microtype}
  \UseMicrotypeSet[protrusion]{basicmath} % disable protrusion for tt fonts
}{}
\makeatletter
\@ifundefined{KOMAClassName}{% if non-KOMA class
  \IfFileExists{parskip.sty}{%
    \usepackage{parskip}
  }{% else
    \setlength{\parindent}{0pt}
    \setlength{\parskip}{6pt plus 2pt minus 1pt}}
}{% if KOMA class
  \KOMAoptions{parskip=half}}
\makeatother
\usepackage{xcolor}
\usepackage[margin=1in]{geometry}
\usepackage{graphicx}
\makeatletter
\newsavebox\pandoc@box
\newcommand*\pandocbounded[1]{% scales image to fit in text height/width
  \sbox\pandoc@box{#1}%
  \Gscale@div\@tempa{\textheight}{\dimexpr\ht\pandoc@box+\dp\pandoc@box\relax}%
  \Gscale@div\@tempb{\linewidth}{\wd\pandoc@box}%
  \ifdim\@tempb\p@<\@tempa\p@\let\@tempa\@tempb\fi% select the smaller of both
  \ifdim\@tempa\p@<\p@\scalebox{\@tempa}{\usebox\pandoc@box}%
  \else\usebox{\pandoc@box}%
  \fi%
}
% Set default figure placement to htbp
\def\fps@figure{htbp}
\makeatother
\setlength{\emergencystretch}{3em} % prevent overfull lines
\providecommand{\tightlist}{%
  \setlength{\itemsep}{0pt}\setlength{\parskip}{0pt}}
\setcounter{secnumdepth}{5}
\usepackage{bookmark}
\IfFileExists{xurl.sty}{\usepackage{xurl}}{} % add URL line breaks if available
\urlstyle{same}
\hypersetup{
  pdftitle={Coleta e Classificação de Discursos do Senado Federal utilizando Agent AI},
  pdfauthor={Ana Beatriz; Alberto Bastos; Victor Caetano},
  hidelinks,
  pdfcreator={LaTeX via pandoc}}

\title{Coleta e Classificação de Discursos do Senado Federal utilizando
Agent AI}
\author{Ana Beatriz \and Alberto Bastos \and Victor Caetano}
\date{2025-05-14}

\begin{document}
\maketitle

{
\setcounter{tocdepth}{5}
\tableofcontents
}
\begin{abstract}
Este é o resumo do artigo. Deve conter uma breve descrição dos objetivos, métodos, resultados e conclusões. Máximo de 250 palavras.
\end{abstract}

\textbf{Palavras-chave:} palavra1; palavra2; palavra3

\section{Introdução}

Contextualização do problema, lacunas na literatura, objetivos.

\section{Trabalhos Relacionados}

Discussão de estudos anteriores relevantes, comparando abordagens,
lacunas e contribuições.

\section{Metodologia}

Descrição clara e replicável dos métodos utilizados, incluindo
materiais, ferramentas e processos.

\section{Resultados}

Apresentação dos dados e análise estatística ou qualitativa.

\section{Discussão}

Interpretação dos resultados, relação com hipóteses, implicações
teóricas ou práticas.

\subsection{Ameaças à Validade}

Discussão das possíveis limitações do estudo quanto à validade interna,
externa, de construção e de conclusão.

\subsection{Trabalhos Futuros}

\colorbox{yellow}{Futuramente outras implementações podem ser feitas no código, de modo a permitir que funcionalidade, após interação com usuário, sugira temas.}

\section{Conclusão}

Resumo dos principais achados, contribuições e encerramento do artigo.

\section*{Agradecimentos}

(Se aplicável)

\bibliographystyle{apalike}
\bibliography{referencias}

\appendix
\section*{Apêndice}

(Material suplementar, se necessário)

\textbackslash end\{document\}

\end{document}
