% Options for packages loaded elsewhere
\PassOptionsToPackage{unicode}{hyperref}
\PassOptionsToPackage{hyphens}{url}
%
\documentclass[
]{article}
\usepackage{amsmath,amssymb}
\usepackage{iftex}
\ifPDFTeX
  \usepackage[T1]{fontenc}
  \usepackage[utf8]{inputenc}
  \usepackage{textcomp} % provide euro and other symbols
\else % if luatex or xetex
  \usepackage{unicode-math} % this also loads fontspec
  \defaultfontfeatures{Scale=MatchLowercase}
  \defaultfontfeatures[\rmfamily]{Ligatures=TeX,Scale=1}
\fi
\usepackage{lmodern}
\ifPDFTeX\else
  % xetex/luatex font selection
\fi
% Use upquote if available, for straight quotes in verbatim environments
\IfFileExists{upquote.sty}{\usepackage{upquote}}{}
\IfFileExists{microtype.sty}{% use microtype if available
  \usepackage[]{microtype}
  \UseMicrotypeSet[protrusion]{basicmath} % disable protrusion for tt fonts
}{}
\makeatletter
\@ifundefined{KOMAClassName}{% if non-KOMA class
  \IfFileExists{parskip.sty}{%
    \usepackage{parskip}
  }{% else
    \setlength{\parindent}{0pt}
    \setlength{\parskip}{6pt plus 2pt minus 1pt}}
}{% if KOMA class
  \KOMAoptions{parskip=half}}
\makeatother
\usepackage{xcolor}
\usepackage[margin=1in]{geometry}
\usepackage{color}
\usepackage{fancyvrb}
\newcommand{\VerbBar}{|}
\newcommand{\VERB}{\Verb[commandchars=\\\{\}]}
\DefineVerbatimEnvironment{Highlighting}{Verbatim}{commandchars=\\\{\}}
% Add ',fontsize=\small' for more characters per line
\usepackage{framed}
\definecolor{shadecolor}{RGB}{248,248,248}
\newenvironment{Shaded}{\begin{snugshade}}{\end{snugshade}}
\newcommand{\AlertTok}[1]{\textcolor[rgb]{0.94,0.16,0.16}{#1}}
\newcommand{\AnnotationTok}[1]{\textcolor[rgb]{0.56,0.35,0.01}{\textbf{\textit{#1}}}}
\newcommand{\AttributeTok}[1]{\textcolor[rgb]{0.13,0.29,0.53}{#1}}
\newcommand{\BaseNTok}[1]{\textcolor[rgb]{0.00,0.00,0.81}{#1}}
\newcommand{\BuiltInTok}[1]{#1}
\newcommand{\CharTok}[1]{\textcolor[rgb]{0.31,0.60,0.02}{#1}}
\newcommand{\CommentTok}[1]{\textcolor[rgb]{0.56,0.35,0.01}{\textit{#1}}}
\newcommand{\CommentVarTok}[1]{\textcolor[rgb]{0.56,0.35,0.01}{\textbf{\textit{#1}}}}
\newcommand{\ConstantTok}[1]{\textcolor[rgb]{0.56,0.35,0.01}{#1}}
\newcommand{\ControlFlowTok}[1]{\textcolor[rgb]{0.13,0.29,0.53}{\textbf{#1}}}
\newcommand{\DataTypeTok}[1]{\textcolor[rgb]{0.13,0.29,0.53}{#1}}
\newcommand{\DecValTok}[1]{\textcolor[rgb]{0.00,0.00,0.81}{#1}}
\newcommand{\DocumentationTok}[1]{\textcolor[rgb]{0.56,0.35,0.01}{\textbf{\textit{#1}}}}
\newcommand{\ErrorTok}[1]{\textcolor[rgb]{0.64,0.00,0.00}{\textbf{#1}}}
\newcommand{\ExtensionTok}[1]{#1}
\newcommand{\FloatTok}[1]{\textcolor[rgb]{0.00,0.00,0.81}{#1}}
\newcommand{\FunctionTok}[1]{\textcolor[rgb]{0.13,0.29,0.53}{\textbf{#1}}}
\newcommand{\ImportTok}[1]{#1}
\newcommand{\InformationTok}[1]{\textcolor[rgb]{0.56,0.35,0.01}{\textbf{\textit{#1}}}}
\newcommand{\KeywordTok}[1]{\textcolor[rgb]{0.13,0.29,0.53}{\textbf{#1}}}
\newcommand{\NormalTok}[1]{#1}
\newcommand{\OperatorTok}[1]{\textcolor[rgb]{0.81,0.36,0.00}{\textbf{#1}}}
\newcommand{\OtherTok}[1]{\textcolor[rgb]{0.56,0.35,0.01}{#1}}
\newcommand{\PreprocessorTok}[1]{\textcolor[rgb]{0.56,0.35,0.01}{\textit{#1}}}
\newcommand{\RegionMarkerTok}[1]{#1}
\newcommand{\SpecialCharTok}[1]{\textcolor[rgb]{0.81,0.36,0.00}{\textbf{#1}}}
\newcommand{\SpecialStringTok}[1]{\textcolor[rgb]{0.31,0.60,0.02}{#1}}
\newcommand{\StringTok}[1]{\textcolor[rgb]{0.31,0.60,0.02}{#1}}
\newcommand{\VariableTok}[1]{\textcolor[rgb]{0.00,0.00,0.00}{#1}}
\newcommand{\VerbatimStringTok}[1]{\textcolor[rgb]{0.31,0.60,0.02}{#1}}
\newcommand{\WarningTok}[1]{\textcolor[rgb]{0.56,0.35,0.01}{\textbf{\textit{#1}}}}
\usepackage{graphicx}
\makeatletter
\newsavebox\pandoc@box
\newcommand*\pandocbounded[1]{% scales image to fit in text height/width
  \sbox\pandoc@box{#1}%
  \Gscale@div\@tempa{\textheight}{\dimexpr\ht\pandoc@box+\dp\pandoc@box\relax}%
  \Gscale@div\@tempb{\linewidth}{\wd\pandoc@box}%
  \ifdim\@tempb\p@<\@tempa\p@\let\@tempa\@tempb\fi% select the smaller of both
  \ifdim\@tempa\p@<\p@\scalebox{\@tempa}{\usebox\pandoc@box}%
  \else\usebox{\pandoc@box}%
  \fi%
}
% Set default figure placement to htbp
\def\fps@figure{htbp}
\makeatother
\setlength{\emergencystretch}{3em} % prevent overfull lines
\providecommand{\tightlist}{%
  \setlength{\itemsep}{0pt}\setlength{\parskip}{0pt}}
\setcounter{secnumdepth}{5}
\usepackage{bookmark}
\IfFileExists{xurl.sty}{\usepackage{xurl}}{} % add URL line breaks if available
\urlstyle{same}
\hypersetup{
  pdftitle={apresentacao\_ajustada},
  hidelinks,
  pdfcreator={LaTeX via pandoc}}

\title{apresentacao\_ajustada}
\author{}
\date{\vspace{-2.5em}2025-05-15}

\begin{document}
\maketitle

{
\setcounter{tocdepth}{5}
\tableofcontents
}
\pagebreak

\begin{center}
UNIVERSIDADE FEDERAL DE SERGIPE
\end{center}
\begin{center}
PROGRAMA DE PÓS-GRADUAÇÃO EM CIÊNCIA DA COMPUTAÇÃO
\end{center}
\hfill\break
\hfill\break
\hfill\break
\hfill\break
\hfill\break
\hfill\break
\hfill\break
\hfill\break
\hfill\break
\hfill\break
\hfill\break
\hfill\break
\begin{center}
TÓPICOS AVANÇADOS EM ENGENHARIA DE SOFTWARE E SISTEMAS DE INFORMAÇÃO I
\end{center}
\begin{center}
\textbf{Atividade 3}
\end{center}
\hfill\break
\hfill\break
\begin{center}
Alberto Bastos\\
Ana Beatriz\\
Victor Caetano\\
\end{center}
\hfill\break
\hfill\break
\hfill\break
\hfill\break
\hfill\break
\hfill\break
\hfill\break
\hfill\break
\hfill\break
\hfill\break
\hfill\break
\hfill\break
\hfill\break
\hfill\break
\hfill\break
\hfill\break
\hfill\break
\hfill\break
\begin{center}
Abril 2025
\end{center}

\pagebreak

\section{Introdução}\label{introduuxe7uxe3o}

Este relatório descreve a evolução de um código Python para coleta de
discursos do Senado Federal via API, demonstrando como foi aprimorado
com o uso do GitHub Copilot, uma ferramenta baseada em inteligência
artificial.

\section{Código Original}\label{cuxf3digo-original}

A primeira versão do código realizava a extração de discursos entre as
datas de 01/03/2025 e 01/04/2025, utilizando a API de dados abertos do
Senado Federal. Ele recuperava campos como:

\begin{itemize}
\tightlist
\item
  Resumo
\item
  Autor
\item
  Partido
\item
  Tipo de uso da palavra
\item
  Link para o texto completo
\end{itemize}

O código utilizava as bibliotecas \texttt{requests} e
\texttt{xml.etree.elementTree} para fazer as requisições e parse do XML.

\begin{Shaded}
\begin{Highlighting}[]
\ImportTok{import}\NormalTok{ requests}
\ImportTok{import}\NormalTok{ xml.etree.ElementTree }\ImportTok{as}\NormalTok{ ET}

\KeywordTok{def}\NormalTok{ buscar\_discursos(data\_inicio, data\_fim):}
\NormalTok{    url }\OperatorTok{=} \SpecialStringTok{f"https://legis.senado.leg.br/dadosabertos/plenario/lista/discursos/}\SpecialCharTok{\{}\NormalTok{data\_inicio}\SpecialCharTok{\}}\SpecialStringTok{/}\SpecialCharTok{\{}\NormalTok{data\_fim}\SpecialCharTok{\}}\SpecialStringTok{"}
\NormalTok{    headers }\OperatorTok{=}\NormalTok{ \{}
        \StringTok{"Accept"}\NormalTok{: }\StringTok{"application/xml"}
\NormalTok{    \}}
\NormalTok{    response }\OperatorTok{=}\NormalTok{ requests.get(url, headers}\OperatorTok{=}\NormalTok{headers)}
    \CommentTok{\# Exibe o status HTTP e o conteúdo completo da resposta para depuração}
    \BuiltInTok{print}\NormalTok{(}\SpecialStringTok{f"Status HTTP: }\SpecialCharTok{\{}\NormalTok{response}\SpecialCharTok{.}\NormalTok{status\_code}\SpecialCharTok{\}}\SpecialStringTok{"}\NormalTok{)}
    \CommentTok{\# print(f"Resposta Completa:\textbackslash{}n\{response.text\}")  \# Descomente se precisar de mais detalhes}
    \ControlFlowTok{if}\NormalTok{ response.status\_code }\OperatorTok{==} \DecValTok{200}\NormalTok{:}
        \ControlFlowTok{try}\NormalTok{:}
\NormalTok{            root }\OperatorTok{=}\NormalTok{ ET.fromstring(response.content)}
            \CommentTok{\# Vamos procurar pelos Pronunciamentos dentro das Sessões}
\NormalTok{            discursos }\OperatorTok{=}\NormalTok{ []}
            \ControlFlowTok{for}\NormalTok{ sessao }\KeywordTok{in}\NormalTok{ root.findall(}\StringTok{\textquotesingle{}.//Sessoes//Sessao//Pronunciamentos//Pronunciamento\textquotesingle{}}\NormalTok{):}
                \CommentTok{\# Acessar os valores de forma segura}
\NormalTok{                codigo\_pronunciamento }\OperatorTok{=}\NormalTok{ sessao.find(}\StringTok{\textquotesingle{}CodigoPronunciamento\textquotesingle{}}\NormalTok{)}
\NormalTok{                tipo\_uso\_palavra\_codigo }\OperatorTok{=}\NormalTok{ sessao.find(}\StringTok{\textquotesingle{}.//TipoUsoPalavra/Codigo\textquotesingle{}}\NormalTok{)}
\NormalTok{                tipo\_uso\_palavra\_descricao }\OperatorTok{=}\NormalTok{ sessao.find(}\StringTok{\textquotesingle{}.//TipoUsoPalavra/Descricao\textquotesingle{}}\NormalTok{)}
\NormalTok{                resumo }\OperatorTok{=}\NormalTok{ sessao.find(}\StringTok{\textquotesingle{}Resumo\textquotesingle{}}\NormalTok{)}
\NormalTok{                texto\_integral }\OperatorTok{=}\NormalTok{ sessao.find(}\StringTok{\textquotesingle{}TextoIntegralTxt\textquotesingle{}}\NormalTok{)}
\NormalTok{                url\_texto\_binario }\OperatorTok{=}\NormalTok{ sessao.find(}\StringTok{\textquotesingle{}UrlTextoBinario\textquotesingle{}}\NormalTok{)}
                \CommentTok{\# Acessar dados do autor}
\NormalTok{                nome\_autor }\OperatorTok{=}\NormalTok{ sessao.find(}\StringTok{\textquotesingle{}NomeAutor\textquotesingle{}}\NormalTok{)}
\NormalTok{                partido }\OperatorTok{=}\NormalTok{ sessao.find(}\StringTok{\textquotesingle{}Partido\textquotesingle{}}\NormalTok{)}
                \CommentTok{\# Adicionando os dados ao dicionário com verificações de \textquotesingle{}None\textquotesingle{}}
\NormalTok{                discurso\_info }\OperatorTok{=}\NormalTok{ \{}
                    \StringTok{\textquotesingle{}CodigoPronunciamento\textquotesingle{}}\NormalTok{: codigo\_pronunciamento.text }\ControlFlowTok{if}\NormalTok{ codigo\_pronunciamento }\KeywordTok{is} \KeywordTok{not} \VariableTok{None} \ControlFlowTok{else} \StringTok{\textquotesingle{}Não disponível\textquotesingle{}}\NormalTok{,}
                    \StringTok{\textquotesingle{}TipoUsoPalavra\textquotesingle{}}\NormalTok{: \{}
                        \StringTok{\textquotesingle{}Codigo\textquotesingle{}}\NormalTok{: tipo\_uso\_palavra\_codigo.text }\ControlFlowTok{if}\NormalTok{ tipo\_uso\_palavra\_codigo }\KeywordTok{is} \KeywordTok{not} \VariableTok{None} \ControlFlowTok{else} \StringTok{\textquotesingle{}Não disponível\textquotesingle{}}\NormalTok{,}
                        \StringTok{\textquotesingle{}Descricao\textquotesingle{}}\NormalTok{: tipo\_uso\_palavra\_descricao.text }\ControlFlowTok{if}\NormalTok{ tipo\_uso\_palavra\_descricao }\KeywordTok{is} \KeywordTok{not} \VariableTok{None} \ControlFlowTok{else} \StringTok{\textquotesingle{}Não disponível\textquotesingle{}}\NormalTok{,}
\NormalTok{                    \},}
                    \StringTok{\textquotesingle{}Resumo\textquotesingle{}}\NormalTok{: resumo.text }\ControlFlowTok{if}\NormalTok{ resumo }\KeywordTok{is} \KeywordTok{not} \VariableTok{None} \ControlFlowTok{else} \StringTok{\textquotesingle{}Não disponível\textquotesingle{}}\NormalTok{,}
                    \StringTok{\textquotesingle{}TextoIntegral\textquotesingle{}}\NormalTok{: texto\_integral.text }\ControlFlowTok{if}\NormalTok{ texto\_integral }\KeywordTok{is} \KeywordTok{not} \VariableTok{None} \ControlFlowTok{else} \StringTok{\textquotesingle{}Não disponível\textquotesingle{}}\NormalTok{,}
                    \StringTok{\textquotesingle{}UrlTextoBinario\textquotesingle{}}\NormalTok{: url\_texto\_binario.text }\ControlFlowTok{if}\NormalTok{ url\_texto\_binario }\KeywordTok{is} \KeywordTok{not} \VariableTok{None} \ControlFlowTok{else} \StringTok{\textquotesingle{}Não disponível\textquotesingle{}}\NormalTok{,}
                    \StringTok{\textquotesingle{}NomeAutor\textquotesingle{}}\NormalTok{: nome\_autor.text }\ControlFlowTok{if}\NormalTok{ nome\_autor }\KeywordTok{is} \KeywordTok{not} \VariableTok{None} \ControlFlowTok{else} \StringTok{\textquotesingle{}Não disponível\textquotesingle{}}\NormalTok{,}
                    \StringTok{\textquotesingle{}Partido\textquotesingle{}}\NormalTok{: partido.text }\ControlFlowTok{if}\NormalTok{ partido }\KeywordTok{is} \KeywordTok{not} \VariableTok{None} \ControlFlowTok{else} \StringTok{\textquotesingle{}Não disponível\textquotesingle{}}
\NormalTok{                \}}
\NormalTok{                discursos.append(discurso\_info)}
            \ControlFlowTok{return}\NormalTok{ discursos  }\CommentTok{\# Retorna a lista de discursos}
        \ControlFlowTok{except}\NormalTok{ ET.ParseError }\ImportTok{as}\NormalTok{ e:}
            \BuiltInTok{print}\NormalTok{(}\StringTok{"Erro ao fazer o parse do XML:"}\NormalTok{, e)}
            \BuiltInTok{print}\NormalTok{(}\StringTok{"Resposta recebida (parcial):"}\NormalTok{, response.text[:}\DecValTok{500}\NormalTok{])}
            \ControlFlowTok{return} \VariableTok{None}
    \ControlFlowTok{else}\NormalTok{:}
        \BuiltInTok{print}\NormalTok{(}\StringTok{"Falha na requisição!"}\NormalTok{)}
        \ControlFlowTok{return} \VariableTok{None}

\CommentTok{\# Exemplo de uso com datas que funcionam}
\NormalTok{discursos }\OperatorTok{=}\NormalTok{ buscar\_discursos(}\StringTok{"20250301"}\NormalTok{, }\StringTok{"20250401"}\NormalTok{)}

\CommentTok{\# Exibindo os discursos de forma legível}
\ControlFlowTok{if}\NormalTok{ discursos:}
    \ControlFlowTok{for}\NormalTok{ i, discurso }\KeywordTok{in} \BuiltInTok{enumerate}\NormalTok{(discursos, }\DecValTok{1}\NormalTok{):}
        \BuiltInTok{print}\NormalTok{(}\SpecialStringTok{f"Discurso }\SpecialCharTok{\{}\NormalTok{i}\SpecialCharTok{\}}\SpecialStringTok{:"}\NormalTok{)}
        \ControlFlowTok{for}\NormalTok{ key, value }\KeywordTok{in}\NormalTok{ discurso.items():}
            \BuiltInTok{print}\NormalTok{(}\SpecialStringTok{f"}\SpecialCharTok{\{}\NormalTok{key}\SpecialCharTok{\}}\SpecialStringTok{: }\SpecialCharTok{\{}\NormalTok{value}\SpecialCharTok{\}}\SpecialStringTok{"}\NormalTok{)}
        \BuiltInTok{print}\NormalTok{(}\StringTok{"="}\OperatorTok{*}\DecValTok{40}\NormalTok{)}
\ControlFlowTok{else}\NormalTok{:}
    \BuiltInTok{print}\NormalTok{(}\StringTok{"Não foi possível recuperar os discursos."}\NormalTok{)}
\end{Highlighting}
\end{Shaded}

\begin{verbatim}
Status HTTP: 200
Discurso 1:
CodigoPronunciamento: 512673
TipoUsoPalavra: {'Codigo': '4819', 'Descricao': 'Discurso'}
Resumo: Exposição sobre a importância da regulamentação da reforma tributária, da fiscalização da execução de políticas públicas, a exemplo da Lei Geral de Saneamento, e da regularização de terras no Brasil. Críticas ao excesso de burocracia da Administração Pública e à polarização política que, segundo S. Exa., compromete o avanço institucional.
TextoIntegral: https://legis.senado.leg.br/dadosabertos/discurso/texto-integral/512673
UrlTextoBinario: https://legis.senado.leg.br/dadosabertos/discurso/texto-binario/512673
NomeAutor: Confúcio Moura
Partido: MDB
========================================
Discurso 2:
CodigoPronunciamento: 512672
TipoUsoPalavra: {'Codigo': '4819', 'Descricao': 'Discurso'}
Resumo: Críticas ao STF pelo suposto uso de processos judiciais como instrumento de pressão política. Defesa da anistia aos presos pelos atos de 8 de janeiro de 2023. Prestação de contas e registro da doação do salário de S. Exa. à Associação dos Familiares e Vítimas do 08 de Janeiro (Asfav).
TextoIntegral: https://legis.senado.leg.br/dadosabertos/discurso/texto-integral/512672
UrlTextoBinario: https://legis.senado.leg.br/dadosabertos/discurso/texto-binario/512672
NomeAutor: Eduardo Girão
Partido: NOVO
========================================
Discurso 3:
CodigoPronunciamento: 512670
TipoUsoPalavra: {'Codigo': '4819', 'Descricao': 'Discurso'}
Resumo: Defesa da redução da jornada de trabalho sem redução salarial, e registro da tramitação, na CCJ, da PEC 148/2015, da qual S. Exa. é o primeiro signatário. Breve histórico dessa pauta no Brasil e no mundo.
TextoIntegral: https://legis.senado.leg.br/dadosabertos/discurso/texto-integral/512670
UrlTextoBinario: https://legis.senado.leg.br/dadosabertos/discurso/texto-binario/512670
NomeAutor: Paulo Paim
Partido: PT
======================================== 
\end{verbatim}

\section{Motivação para Melhoria}\label{motivauxe7uxe3o-para-melhoria}

Durante o uso do script, observamos que o campo Resumo trazia uma versão
condensada do conteúdo dos discursos. Isso inspirou a ideia de aplicar
classificação automática de temas com base nesse resumo, permitindo
analisar tendências temáticas sem intervenção manual.

\section{Uso do GitHub Copilot}\label{uso-do-github-copilot}

Para que fosse possível classificar sem sair do VSCode pedimos ao Agent
AI Copilot com o seguinte comando:

\begin{quote}
com base nos retornos dos discursos do senado, existe o campo resumo em
que possui o resumo do discurso, com base nesse resumo voce é capaz de
classificar o tema?
\end{quote}

O Copilot sugeriu o uso da biblioteca \texttt{transformers} com o
\emph{pipeline} \texttt{zero-shot-classification} e o modelo
\texttt{facebook/bart-large-mnli}.

\begin{figure}
\centering
\pandocbounded{\includegraphics[keepaspectratio]{/home/bastos/Downloads/github_copilot.png}}
\caption{Agent AI Copilot}
\end{figure}

Essa sugestão foi implementada para classificar os discursos em temas
como: \textbf{Política, Economia, Educação, Saúde, Meio Ambiente,
Tecnologia e Segurança}.

\begin{Shaded}
\begin{Highlighting}[]
\ImportTok{import}\NormalTok{ requests}
\ImportTok{import}\NormalTok{ xml.etree.ElementTree }\ImportTok{as}\NormalTok{ ET}
\ImportTok{from}\NormalTok{ transformers }\ImportTok{import}\NormalTok{ pipeline}

\KeywordTok{def}\NormalTok{ classificar\_tema(resumo):   }
\NormalTok{    classificador }\OperatorTok{=}\NormalTok{ pipeline(}\StringTok{"zero{-}shot{-}classification"}\NormalTok{, model}\OperatorTok{=}\StringTok{"facebook/bart{-}large{-}mnli"}\NormalTok{)}
\NormalTok{    temas }\OperatorTok{=}\NormalTok{ [}\StringTok{"Política"}\NormalTok{, }\StringTok{"Economia"}\NormalTok{, }\StringTok{"Educação"}\NormalTok{, }\StringTok{"Saúde"}\NormalTok{, }\StringTok{"Meio Ambiente"}\NormalTok{, }\StringTok{"Tecnologia"}\NormalTok{, }\StringTok{"Segurança"}\NormalTok{]    }
\NormalTok{    resultado }\OperatorTok{=}\NormalTok{ classificador(resumo, temas)   }
    \ControlFlowTok{return}\NormalTok{ resultado[}\StringTok{\textquotesingle{}labels\textquotesingle{}}\NormalTok{][}\DecValTok{0}\NormalTok{]}

\KeywordTok{def}\NormalTok{ buscar\_discursos(data\_inicio, data\_fim):}
\NormalTok{    url }\OperatorTok{=} \SpecialStringTok{f"https://legis.senado.leg.br/dadosabertos/plenario/lista/discursos/}\SpecialCharTok{\{}\NormalTok{data\_inicio}\SpecialCharTok{\}}\SpecialStringTok{/}\SpecialCharTok{\{}\NormalTok{data\_fim}\SpecialCharTok{\}}\SpecialStringTok{"}

\NormalTok{    headers }\OperatorTok{=}\NormalTok{ \{}
        \StringTok{"Accept"}\NormalTok{: }\StringTok{"application/xml"}
\NormalTok{    \}}
\NormalTok{    response }\OperatorTok{=}\NormalTok{ requests.get(url, headers}\OperatorTok{=}\NormalTok{headers)}
    \BuiltInTok{print}\NormalTok{(}\SpecialStringTok{f"Status HTTP: }\SpecialCharTok{\{}\NormalTok{response}\SpecialCharTok{.}\NormalTok{status\_code}\SpecialCharTok{\}}\SpecialStringTok{"}\NormalTok{)}
    \ControlFlowTok{if}\NormalTok{ response.status\_code }\OperatorTok{==} \DecValTok{200}\NormalTok{:}
        \ControlFlowTok{try}\NormalTok{:}
\NormalTok{            root }\OperatorTok{=}\NormalTok{ ET.fromstring(response.content)}
\NormalTok{            discursos }\OperatorTok{=}\NormalTok{ []}
            \ControlFlowTok{for}\NormalTok{ sessao }\KeywordTok{in}\NormalTok{ root.findall(}\StringTok{\textquotesingle{}.//Sessoes//Sessao//Pronunciamentos//Pronunciamento\textquotesingle{}}\NormalTok{):              }
\NormalTok{                codigo\_pronunciamento }\OperatorTok{=}\NormalTok{ sessao.find(}\StringTok{\textquotesingle{}CodigoPronunciamento\textquotesingle{}}\NormalTok{)}
\NormalTok{                tipo\_uso\_palavra\_codigo }\OperatorTok{=}\NormalTok{ sessao.find(}\StringTok{\textquotesingle{}.//TipoUsoPalavra/Codigo\textquotesingle{}}\NormalTok{)}
\NormalTok{                tipo\_uso\_palavra\_descricao }\OperatorTok{=}\NormalTok{ sessao.find(}\StringTok{\textquotesingle{}.//TipoUsoPalavra/Descricao\textquotesingle{}}\NormalTok{)}
\NormalTok{                resumo }\OperatorTok{=}\NormalTok{ sessao.find(}\StringTok{\textquotesingle{}Resumo\textquotesingle{}}\NormalTok{)}
\NormalTok{                texto\_integral }\OperatorTok{=}\NormalTok{ sessao.find(}\StringTok{\textquotesingle{}TextoIntegralTxt\textquotesingle{}}\NormalTok{)}
\NormalTok{                url\_texto\_binario }\OperatorTok{=}\NormalTok{ sessao.find(}\StringTok{\textquotesingle{}UrlTextoBinario\textquotesingle{}}\NormalTok{)       }
\NormalTok{                nome\_autor }\OperatorTok{=}\NormalTok{ sessao.find(}\StringTok{\textquotesingle{}NomeAutor\textquotesingle{}}\NormalTok{)}
\NormalTok{                partido }\OperatorTok{=}\NormalTok{ sessao.find(}\StringTok{\textquotesingle{}Partido\textquotesingle{}}\NormalTok{)}
\NormalTok{                discurso\_info }\OperatorTok{=}\NormalTok{ \{}
                    \StringTok{\textquotesingle{}CodigoPronunciamento\textquotesingle{}}\NormalTok{: codigo\_pronunciamento.text }\ControlFlowTok{if}\NormalTok{ codigo\_pronunciamento }\KeywordTok{is} \KeywordTok{not} \VariableTok{None} \ControlFlowTok{else} \StringTok{\textquotesingle{}Não disponível\textquotesingle{}}\NormalTok{,}
                    \StringTok{\textquotesingle{}TipoUsoPalavra\textquotesingle{}}\NormalTok{: \{}
                        \StringTok{\textquotesingle{}Codigo\textquotesingle{}}\NormalTok{: tipo\_uso\_palavra\_codigo.text }\ControlFlowTok{if}\NormalTok{ tipo\_uso\_palavra\_codigo }\KeywordTok{is} \KeywordTok{not} \VariableTok{None} \ControlFlowTok{else} \StringTok{\textquotesingle{}Não disponível\textquotesingle{}}\NormalTok{,}
                        \StringTok{\textquotesingle{}Descricao\textquotesingle{}}\NormalTok{: tipo\_uso\_palavra\_descricao.text }\ControlFlowTok{if}\NormalTok{ tipo\_uso\_palavra\_descricao }\KeywordTok{is} \KeywordTok{not} \VariableTok{None} \ControlFlowTok{else} \StringTok{\textquotesingle{}Não disponível\textquotesingle{}}\NormalTok{,}
\NormalTok{                    \},}
                    \StringTok{\textquotesingle{}Resumo\textquotesingle{}}\NormalTok{: resumo.text }\ControlFlowTok{if}\NormalTok{ resumo }\KeywordTok{is} \KeywordTok{not} \VariableTok{None} \ControlFlowTok{else} \StringTok{\textquotesingle{}Não disponível\textquotesingle{}}\NormalTok{,}
                    \StringTok{\textquotesingle{}TextoIntegral\textquotesingle{}}\NormalTok{: texto\_integral.text }\ControlFlowTok{if}\NormalTok{ texto\_integral }\KeywordTok{is} \KeywordTok{not} \VariableTok{None} \ControlFlowTok{else} \StringTok{\textquotesingle{}Não disponível\textquotesingle{}}\NormalTok{,}
                    \StringTok{\textquotesingle{}UrlTextoBinario\textquotesingle{}}\NormalTok{: url\_texto\_binario.text }\ControlFlowTok{if}\NormalTok{ url\_texto\_binario }\KeywordTok{is} \KeywordTok{not} \VariableTok{None} \ControlFlowTok{else} \StringTok{\textquotesingle{}Não disponível\textquotesingle{}}\NormalTok{,}
                    \StringTok{\textquotesingle{}NomeAutor\textquotesingle{}}\NormalTok{: nome\_autor.text }\ControlFlowTok{if}\NormalTok{ nome\_autor }\KeywordTok{is} \KeywordTok{not} \VariableTok{None} \ControlFlowTok{else} \StringTok{\textquotesingle{}Não disponível\textquotesingle{}}\NormalTok{,}
                    \StringTok{\textquotesingle{}Partido\textquotesingle{}}\NormalTok{: partido.text }\ControlFlowTok{if}\NormalTok{ partido }\KeywordTok{is} \KeywordTok{not} \VariableTok{None} \ControlFlowTok{else} \StringTok{\textquotesingle{}Não disponível\textquotesingle{}}\NormalTok{,}
                    \CommentTok{\# Classifica o tema com base no resumo}
                    \StringTok{\textquotesingle{}Tema\textquotesingle{}}\NormalTok{: classificar\_tema(resumo.text) }\ControlFlowTok{if}\NormalTok{ resumo }\KeywordTok{is} \KeywordTok{not} \VariableTok{None} \ControlFlowTok{else} \StringTok{\textquotesingle{}Não disponível\textquotesingle{}}
\NormalTok{                \}}

\NormalTok{                discursos.append(discurso\_info)}
            \ControlFlowTok{return}\NormalTok{ discursos  }\CommentTok{\# Retorna a lista de discursos}
        \ControlFlowTok{except}\NormalTok{ ET.ParseError }\ImportTok{as}\NormalTok{ e:}
            \BuiltInTok{print}\NormalTok{(}\StringTok{"Erro ao fazer o parse do XML:"}\NormalTok{, e)}
            \BuiltInTok{print}\NormalTok{(}\StringTok{"Resposta recebida (parcial):"}\NormalTok{, response.text[:}\DecValTok{500}\NormalTok{])}
            \ControlFlowTok{return} \VariableTok{None}
    \ControlFlowTok{else}\NormalTok{:}
        \BuiltInTok{print}\NormalTok{(}\StringTok{"Falha na requisição!"}\NormalTok{)}
        \ControlFlowTok{return} \VariableTok{None}

\CommentTok{\# Exemplo de uso com datas que funcionam}
\NormalTok{discursos }\OperatorTok{=}\NormalTok{ buscar\_discursos(}\StringTok{"20250301"}\NormalTok{, }\StringTok{"20250401"}\NormalTok{)}

\CommentTok{\# Exibindo os discursos de forma legível}
\ControlFlowTok{if}\NormalTok{ discursos:}
    \ControlFlowTok{for}\NormalTok{ i, discurso }\KeywordTok{in} \BuiltInTok{enumerate}\NormalTok{(discursos, }\DecValTok{1}\NormalTok{):}
        \BuiltInTok{print}\NormalTok{(}\SpecialStringTok{f"Discurso }\SpecialCharTok{\{}\NormalTok{i}\SpecialCharTok{\}}\SpecialStringTok{:"}\NormalTok{)}
        \ControlFlowTok{for}\NormalTok{ key, value }\KeywordTok{in}\NormalTok{ discurso.items():}
            \BuiltInTok{print}\NormalTok{(}\SpecialStringTok{f"}\SpecialCharTok{\{}\NormalTok{key}\SpecialCharTok{\}}\SpecialStringTok{: }\SpecialCharTok{\{}\NormalTok{value}\SpecialCharTok{\}}\SpecialStringTok{"}\NormalTok{)}
        \BuiltInTok{print}\NormalTok{(}\StringTok{"="}\OperatorTok{*}\DecValTok{40}\NormalTok{)}
\ControlFlowTok{else}\NormalTok{:}
    \BuiltInTok{print}\NormalTok{(}\StringTok{"Não foi possível recuperar os discursos."}\NormalTok{)}
\end{Highlighting}
\end{Shaded}

\begin{verbatim}
Status HTTP: 200



config.json:   0%|          | 0.00/1.15k [00:00<?, ?B/s]



model.safetensors:   0%|          | 0.00/1.63G [00:00<?, ?B/s]
\end{verbatim}

\section{Exemplos e Resultados}\label{exemplos-e-resultados}

Após a implementação, ao executar o código, os discursos passaram a ser
retornados com seus respectivos temas.

Alguns exemplos da execução:

\begin{figure}
\centering
\pandocbounded{\includegraphics[keepaspectratio]{/home/bastos/Downloads/retorno.png}}
\caption{Resposta API Senado}
\end{figure}

\section{Considerações técnicas}\label{considerauxe7uxf5es-tuxe9cnicas}

Durante a execução inicial do modelo, foi exibida a mensagem
\texttt{Device\ set\ to\ use\ cpu}, indicando que o modelo estava sendo
carregado. Por ser um processo que estava demorando muito resolvemos
questionar ao modelo de Inteligência Artificial GPT se era normal
considerando nosso \emph{script} o tempo que estava levando o processo.
A resposta do GPT

\begin{quote}
Isso significa que agora o TensorFlow foi instalado corretamente e o seu
código está conseguindo carregar o modelo facebook/bart-large-mnli para
fazer a classificação.
\end{quote}

\begin{quote}
Esse trecho aqui é só log do TensorFlow e da Hugging Face dizendo que
tudo está sendo executado na CPU (o que é normal, já que você
provavelmente não está usando uma GPU):
\end{quote}

\begin{quote}
Mas por que está demorando? O modelo facebook/bart-large-mnli é grande
e, na primeira vez, ele precisa ser:
\end{quote}

\begin{quote}
Baixado da Hugging Face Hub (\textasciitilde1.6 GB). Carregado para a
memória. Rodado para cada discurso (o que pode ser lento se houver
muitos).
\end{quote}

\begin{quote}
O que você pode fazer agora: Esperar terminar o processo Se seu código
está ainda rodando e você não recebeu erro, deixe ele completar. Pode
demorar alguns minutos, especialmente se forem muitos discursos ou se
sua internet estiver lenta para baixar o modelo.
\end{quote}

\begin{quote}
Dica de melhoria: carregar o modelo uma vez só! Como falei antes, você
está carregando o modelo dentro da função classificar\_tema a cada
chamada. Isso torna tudo muito mais lento.''
\end{quote}

O código sugerido pelo GPT ficou:

\begin{Shaded}
\begin{Highlighting}[]
\ImportTok{import}\NormalTok{ requests}
\ImportTok{import}\NormalTok{ xml.etree.ElementTree }\ImportTok{as}\NormalTok{ ET}
\ImportTok{from}\NormalTok{ transformers }\ImportTok{import}\NormalTok{ pipeline}

\CommentTok{\#  Carrega o modelo UMA vez só}
\NormalTok{classificador }\OperatorTok{=}\NormalTok{ pipeline(}\StringTok{"zero{-}shot{-}classification"}\NormalTok{, model}\OperatorTok{=}\StringTok{"facebook/bart{-}large{-}mnli"}\NormalTok{)}

\KeywordTok{def}\NormalTok{ classificar\_tema(resumo):}
    \CommentTok{\# Define os possíveis temas para classificação}
\NormalTok{    temas }\OperatorTok{=}\NormalTok{ [}\StringTok{"Política"}\NormalTok{, }\StringTok{"Economia"}\NormalTok{, }\StringTok{"Educação"}\NormalTok{, }\StringTok{"Saúde"}\NormalTok{, }\StringTok{"Meio Ambiente"}\NormalTok{, }\StringTok{"Tecnologia"}\NormalTok{, }\StringTok{"Segurança"}\NormalTok{]}
    \ControlFlowTok{try}\NormalTok{:}
        \CommentTok{\# Realiza a classificação}
\NormalTok{        resultado }\OperatorTok{=}\NormalTok{ classificador(resumo, temas)}
        \ControlFlowTok{return}\NormalTok{ resultado[}\StringTok{\textquotesingle{}labels\textquotesingle{}}\NormalTok{][}\DecValTok{0}\NormalTok{]}
    \ControlFlowTok{except} \PreprocessorTok{Exception} \ImportTok{as}\NormalTok{ e:}
        \BuiltInTok{print}\NormalTok{(}\SpecialStringTok{f"Erro ao classificar tema: }\SpecialCharTok{\{}\NormalTok{e}\SpecialCharTok{\}}\SpecialStringTok{"}\NormalTok{)}
        \ControlFlowTok{return} \StringTok{\textquotesingle{}Erro na classificação\textquotesingle{}}

\KeywordTok{def}\NormalTok{ buscar\_discursos(data\_inicio, data\_fim):}
\NormalTok{    url }\OperatorTok{=} \SpecialStringTok{f"https://legis.senado.leg.br/dadosabertos/plenario/lista/discursos/}\SpecialCharTok{\{}\NormalTok{data\_inicio}\SpecialCharTok{\}}\SpecialStringTok{/}\SpecialCharTok{\{}\NormalTok{data\_fim}\SpecialCharTok{\}}\SpecialStringTok{"}
\NormalTok{    headers }\OperatorTok{=}\NormalTok{ \{}
        \StringTok{"Accept"}\NormalTok{: }\StringTok{"application/xml"}
\NormalTok{    \}}
\NormalTok{    response }\OperatorTok{=}\NormalTok{ requests.get(url, headers}\OperatorTok{=}\NormalTok{headers)}
    \BuiltInTok{print}\NormalTok{(}\SpecialStringTok{f"Status HTTP: }\SpecialCharTok{\{}\NormalTok{response}\SpecialCharTok{.}\NormalTok{status\_code}\SpecialCharTok{\}}\SpecialStringTok{"}\NormalTok{)}
    \ControlFlowTok{if}\NormalTok{ response.status\_code }\OperatorTok{==} \DecValTok{200}\NormalTok{:}
        \ControlFlowTok{try}\NormalTok{:}
\NormalTok{            root }\OperatorTok{=}\NormalTok{ ET.fromstring(response.content)}
\NormalTok{            discursos }\OperatorTok{=}\NormalTok{ []}
            \ControlFlowTok{for}\NormalTok{ sessao }\KeywordTok{in}\NormalTok{ root.findall(}\StringTok{\textquotesingle{}.//Sessoes//Sessao//Pronunciamentos//Pronunciamento\textquotesingle{}}\NormalTok{):}
\NormalTok{                codigo\_pronunciamento }\OperatorTok{=}\NormalTok{ sessao.find(}\StringTok{\textquotesingle{}CodigoPronunciamento\textquotesingle{}}\NormalTok{)}
\NormalTok{                tipo\_uso\_palavra\_codigo }\OperatorTok{=}\NormalTok{ sessao.find(}\StringTok{\textquotesingle{}.//TipoUsoPalavra/Codigo\textquotesingle{}}\NormalTok{)}
\NormalTok{                tipo\_uso\_palavra\_descricao }\OperatorTok{=}\NormalTok{ sessao.find(}\StringTok{\textquotesingle{}.//TipoUsoPalavra/Descricao\textquotesingle{}}\NormalTok{)}
\NormalTok{                resumo }\OperatorTok{=}\NormalTok{ sessao.find(}\StringTok{\textquotesingle{}Resumo\textquotesingle{}}\NormalTok{)}
\NormalTok{                texto\_integral }\OperatorTok{=}\NormalTok{ sessao.find(}\StringTok{\textquotesingle{}TextoIntegralTxt\textquotesingle{}}\NormalTok{)}
\NormalTok{                url\_texto\_binario }\OperatorTok{=}\NormalTok{ sessao.find(}\StringTok{\textquotesingle{}UrlTextoBinario\textquotesingle{}}\NormalTok{)}
\NormalTok{                nome\_autor }\OperatorTok{=}\NormalTok{ sessao.find(}\StringTok{\textquotesingle{}NomeAutor\textquotesingle{}}\NormalTok{)}
\NormalTok{                partido }\OperatorTok{=}\NormalTok{ sessao.find(}\StringTok{\textquotesingle{}Partido\textquotesingle{}}\NormalTok{)}

\NormalTok{                discurso\_info }\OperatorTok{=}\NormalTok{ \{}
                    \StringTok{\textquotesingle{}CodigoPronunciamento\textquotesingle{}}\NormalTok{: codigo\_pronunciamento.text }\ControlFlowTok{if}\NormalTok{ codigo\_pronunciamento }\KeywordTok{is} \KeywordTok{not} \VariableTok{None} \ControlFlowTok{else} \StringTok{\textquotesingle{}Não disponível\textquotesingle{}}\NormalTok{,}
                    \StringTok{\textquotesingle{}TipoUsoPalavra\textquotesingle{}}\NormalTok{: \{}
                        \StringTok{\textquotesingle{}Codigo\textquotesingle{}}\NormalTok{: tipo\_uso\_palavra\_codigo.text }\ControlFlowTok{if}\NormalTok{ tipo\_uso\_palavra\_codigo }\KeywordTok{is} \KeywordTok{not} \VariableTok{None} \ControlFlowTok{else} \StringTok{\textquotesingle{}Não disponível\textquotesingle{}}\NormalTok{,}
                        \StringTok{\textquotesingle{}Descricao\textquotesingle{}}\NormalTok{: tipo\_uso\_palavra\_descricao.text }\ControlFlowTok{if}\NormalTok{ tipo\_uso\_palavra\_descricao }\KeywordTok{is} \KeywordTok{not} \VariableTok{None} \ControlFlowTok{else} \StringTok{\textquotesingle{}Não disponível\textquotesingle{}}\NormalTok{,}
\NormalTok{                    \},}
                    \StringTok{\textquotesingle{}Resumo\textquotesingle{}}\NormalTok{: resumo.text }\ControlFlowTok{if}\NormalTok{ resumo }\KeywordTok{is} \KeywordTok{not} \VariableTok{None} \ControlFlowTok{else} \StringTok{\textquotesingle{}Não disponível\textquotesingle{}}\NormalTok{,}
                    \StringTok{\textquotesingle{}TextoIntegral\textquotesingle{}}\NormalTok{: texto\_integral.text }\ControlFlowTok{if}\NormalTok{ texto\_integral }\KeywordTok{is} \KeywordTok{not} \VariableTok{None} \ControlFlowTok{else} \StringTok{\textquotesingle{}Não disponível\textquotesingle{}}\NormalTok{,}
                    \StringTok{\textquotesingle{}UrlTextoBinario\textquotesingle{}}\NormalTok{: url\_texto\_binario.text }\ControlFlowTok{if}\NormalTok{ url\_texto\_binario }\KeywordTok{is} \KeywordTok{not} \VariableTok{None} \ControlFlowTok{else} \StringTok{\textquotesingle{}Não disponível\textquotesingle{}}\NormalTok{,}
                    \StringTok{\textquotesingle{}NomeAutor\textquotesingle{}}\NormalTok{: nome\_autor.text }\ControlFlowTok{if}\NormalTok{ nome\_autor }\KeywordTok{is} \KeywordTok{not} \VariableTok{None} \ControlFlowTok{else} \StringTok{\textquotesingle{}Não disponível\textquotesingle{}}\NormalTok{,}
                    \StringTok{\textquotesingle{}Partido\textquotesingle{}}\NormalTok{: partido.text }\ControlFlowTok{if}\NormalTok{ partido }\KeywordTok{is} \KeywordTok{not} \VariableTok{None} \ControlFlowTok{else} \StringTok{\textquotesingle{}Não disponível\textquotesingle{}}\NormalTok{,}
                    \StringTok{\textquotesingle{}Tema\textquotesingle{}}\NormalTok{: classificar\_tema(resumo.text) }\ControlFlowTok{if}\NormalTok{ resumo }\KeywordTok{is} \KeywordTok{not} \VariableTok{None} \ControlFlowTok{else} \StringTok{\textquotesingle{}Não disponível\textquotesingle{}}
\NormalTok{                \}}

\NormalTok{                discursos.append(discurso\_info)}
            \ControlFlowTok{return}\NormalTok{ discursos}
        \ControlFlowTok{except}\NormalTok{ ET.ParseError }\ImportTok{as}\NormalTok{ e:}
            \BuiltInTok{print}\NormalTok{(}\StringTok{"Erro ao fazer o parse do XML:"}\NormalTok{, e)}
            \BuiltInTok{print}\NormalTok{(}\StringTok{"Resposta recebida (parcial):"}\NormalTok{, response.text[:}\DecValTok{500}\NormalTok{])}
            \ControlFlowTok{return} \VariableTok{None}
    \ControlFlowTok{else}\NormalTok{:}
        \BuiltInTok{print}\NormalTok{(}\StringTok{"Falha na requisição!"}\NormalTok{)}
        \ControlFlowTok{return} \VariableTok{None}
\CommentTok{\# Teste}
\NormalTok{discursos }\OperatorTok{=}\NormalTok{ buscar\_discursos(}\StringTok{"20250301"}\NormalTok{, }\StringTok{"20250401"}\NormalTok{)}

\ControlFlowTok{if}\NormalTok{ discursos:}
    \ControlFlowTok{for}\NormalTok{ i, discurso }\KeywordTok{in} \BuiltInTok{enumerate}\NormalTok{(discursos, }\DecValTok{1}\NormalTok{):}
        \BuiltInTok{print}\NormalTok{(}\SpecialStringTok{f"Discurso }\SpecialCharTok{\{}\NormalTok{i}\SpecialCharTok{\}}\SpecialStringTok{:"}\NormalTok{)}
        \ControlFlowTok{for}\NormalTok{ key, value }\KeywordTok{in}\NormalTok{ discurso.items():}
            \BuiltInTok{print}\NormalTok{(}\SpecialStringTok{f"}\SpecialCharTok{\{}\NormalTok{key}\SpecialCharTok{\}}\SpecialStringTok{: }\SpecialCharTok{\{}\NormalTok{value}\SpecialCharTok{\}}\SpecialStringTok{"}\NormalTok{)}
        \BuiltInTok{print}\NormalTok{(}\StringTok{"="}\OperatorTok{*}\DecValTok{40}\NormalTok{)}
\ControlFlowTok{else}\NormalTok{:}
    \BuiltInTok{print}\NormalTok{(}\StringTok{"Não foi possível recuperar os discursos."}\NormalTok{)}
\end{Highlighting}
\end{Shaded}

Enquanto o código ajustado pelo Agent AI Copilot levou 508.862 segundos
para trazer os dados e classificá-los, o código foi ajustado pelo GPT
351.253 segundos, indicando um ganho de performance.

\section{Conclusão}\label{conclusuxe3o}

A implementação da IA neste projeto mostrou-se eficaz tanto na aplicação
da classificação temática quanto na otimização da performance do código.

A contribuição do Github Copilot foi importante ao indicar o uso de um
modelo NLP para resolver o problema. No entanto, a sugestão inicial,
apesar de funcional, não foi otimização em termos de desempenho.

Por outro lado, a interação com o ChatGPT mostrou melhorias na estrutura
do código e explicação do motivo do gargalo. A diferença entre as
abordagens mostra como o Copilot atua como um assistente rápido para
quem está no código e o ChatGPT como um consultor estratégico e
explicativo para tratar de assuntos mais técnicos.

\begin{figure}
\centering
\pandocbounded{\includegraphics[keepaspectratio]{/home/bastos/Downloads/qr_code_github.png}}
\caption{QR Code Github}
\end{figure}

\url{https://github.com/AnaBeatriz-Carvalho/AgentAI}

\end{document}
